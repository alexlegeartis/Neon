\documentclass[aspectratio=169]{beamer}

% Theme and basics
\usetheme{Madrid}
\usecolortheme{default}
\usefonttheme{professionalfonts}
\setbeamertemplate{navigation symbols}{}

% Custom footline: short author and short title on all non-title slides
\setbeamertemplate{footline}{%
  \leavevmode
  \hbox{%
    \begin{beamercolorbox}[wd=\paperwidth,ht=2.5ex,dp=1ex,center]{author in head/foot}
      \usebeamerfont{author in head/foot}\insertshortauthor\,\textemdash\,\insertshorttitle
    \end{beamercolorbox}%
  }%
}

% Math and utilities
\usepackage{amsmath,amssymb,amsfonts}
\usepackage{bm}
\usepackage{graphicx}
\usepackage{subcaption}
\usepackage{booktabs}
\usepackage{tabularx}
\usepackage{arydshln}
\usepackage{array}
\usepackage{mathtools}
\usepackage{hyperref}
\usepackage{cleveref}
\usepackage{fontawesome5} % for question mark
%\usepackage[authoryear]{natbib}

% Absolute positioning for overlaying content on the title slide
\usepackage[absolute,overlay]{textpos}

\usepackage{environ}
\newcounter{aequation}
\NewEnviron{aequation}{\refstepcounter{aequation}$$\BODY\eqno{\text{(A\theaequation)}}$$}
\crefname{aequation}{assumption}{assumptions}
\creflabelformat{aequation}{(#2A#1#3)}


% Citations use the standard natbib formatting (as in the article)
\usepackage[style=authoryear, maxcitenames=1, maxbibnames=3, terseinits=true, uniquelist=false, natbib=true,]{biblatex}
\addbibresource{../conf_article/icomp2024_conference.bib}
% Reuse article macros
\input{../conf_article/math_commands.tex}

\newcommand{\norm}[1]{\lVert #1\rVert}
\newcommand{\abs}[1]{\lvert #1 \rvert}

% \epsilon is predefined; override to use \varepsilon
\renewcommand{\epsilon}{\varepsilon}
\newcommand{\Rmn}{\R^{m\times n}}
\newcommand{\cB}{\mathcal{B}}
\newcommand{\cD}{\mathcal{D}}
\newcommand{\cN}{\mathcal{N}}
\newcommand{\cO}{\mathcal{O}}
\newcommand{\cS}{\mathcal{S}}
\newcommand{\Ed}[2]{\mathbb{E}_{#1}\left[#2\right]}
\usepackage{mathtools}
\DeclarePairedDelimiter{\sqne}{\|}{\|_2^2}
%\DeclarePairedDelimiter{\norme}{\|}{\|_2}
\DeclarePairedDelimiter{\normf}{\|}{\|_\mathrm{F}}
\DeclarePairedDelimiter{\normkfk}{\|}{\|_\mathrm{KF-k}}
\DeclarePairedDelimiter{\normfstar}{\|}{\|_\mathrm{F*}}
\DeclarePairedDelimiter{\normftwo}{\|}{\|_\mathrm{F2}}
\DeclarePairedDelimiter{\sqns}{\|}{\|_{\mathrm{op}}^2}
\DeclarePairedDelimiter{\norms}{\|}{\|_{\mathrm{op}}}
\DeclarePairedDelimiter{\sqnn}{\|}{\|_{\mathrm{nuc}}^2}
\DeclarePairedDelimiter{\sqnf}{\|}{\|_{\mathrm{F}}^2}
\DeclarePairedDelimiter{\normn}{\|}{\|_{\mathrm{nuc}}}
\DeclarePairedDelimiter{\normfkfk}{\|}{\|_{\mathrm{F-KF-k}}}
\def\<#1,#2>{\langle #1,#2\rangle}
\DeclarePairedDelimiter{\dotprod}{\langle}{\rangle}

\DeclareMathOperator{\tr}{tr}
\DeclareMathOperator{\diag}{diag}

% Graphics from the article repo
\graphicspath{{../conf_article/figs/}{../figures/13may_muon_neon/}}

% Title
\title[The Ky Fan Norms and Beyond]{The Ky Fan Norms and Beyond: Dual Norms and Combinations for Matrix Optimization}
\author[A. Kravatskiy et al.]{%
  Alexey Kravatskiy\inst{1} \and
  Ivan Kozyrev\inst{1} \and
  Nikolai Kozlov\inst{1} \and
  Alexander Vinogradov\inst{1} \and
  Daniil Merkulov\inst{1,2,3,4} \and
  Ivan Oseledets\inst{5,2}\\[0.3em]
}
\institute[]{%
\inst{1} MIPT \hspace{1em}%
\inst{2} Skoltech \hspace{1em}%
\inst{3} HSE \hspace{1em}%
\inst{4} AI4Science \hspace{1em}%
\inst{5} AIRI
}
\date{\today}

\begin{document}

%-------------------------------------------------------------------------------------
\begin{frame}[plain]
  \titlepage
  % Right-side figure on the title slide
  \begin{textblock*}{0.2\paperwidth}(0.75\paperwidth,0.6\paperheight)
    \includegraphics[width=\linewidth]{fstardual_cifar.pdf}
  \end{textblock*}
  \begin{textblock*}{0.22\paperwidth}(0.05\paperwidth,0.55\paperheight)
    \includegraphics[width=\linewidth]{KyFanDual.pdf}
  \end{textblock*}
\end{frame}
%-------------------------------------------------------------------------------------
\begin{frame}{Moving beyond the spectral norm and Muon}
    \begin{block}{Objective:  $\min_{\mX\in\R^{m\times n}} f(\mX)$}
    \faQuestionCircle \space What will happen if we change the spectral norm in the derivation of Muon?
    \end{block}
    \begin{block}{F-Fanions: Muon, Neon, NSGD, Dion without EF, and so much more}
        From $\norms{\cdot}$ and Muon:
        $$\mX^{t+1} = \mX^{t} - \eta U V^\top$$
        To $\normkfk{\cdot}^\dagger$ and a general F-Fanion:
        $$\mX^{t+1} = \mX^{t} - \eta \left(\alpha \sum_{i=1}^k{u_i v_i^\top} + (1-\alpha)\frac{\mM^t}{\normf{\mM^t}}\right), \alpha \in [0,1]$$
    \end{block}
      
\end{frame}
%-------------------------------------------------------------------------------------
\begin{frame}{Linear Minimization Oracle (LMO)}
% after omitting gamma_k, we obtain:
Let us equip $\Rmn$ with a norm $\norm{\cdot}$. Its dual is $\norm{\mX}^\dagger = \sup_{\norm{\mX'}\leq 1} \<\mX,\mX'>$. $\<\cdot, \cdot>$ is the Frobenius product.
\vspace{0.4em}

LMO leads to the update
$$\mX^{t+1} = \mX^{t} - \eta \arg\max_{\mX \in \cB_1} \<\mM^t, \mX> = \mX^{t} - \eta \{\Delta \in \cB_1 \mid \<\mM^t, \Delta> = \norm{\mM^t}^\dagger\}$$

\vspace{0.4em}

\textbf{Recipe}: It means we seek $\Delta$ from the 1-norm ball that delivers $\<\mM^t, \Delta> = \norm{\mM^t}^\dagger$
\vspace{0.4em}

We will often utilize the SVD of $\mM^t$: $\mM^t = \mU \Sigma \mV^\top$.
\end{frame}
\begin{frame}{Of Matrix and Vector Algorithms}
    %\centering
    %\def\arraystretch{1.1}
    \footnotesize
    \begin{table}
    \caption{lmo optimizers in Schatten $S_p$ norms and in $l_p$ norms.}
    \label{tbl:mat_vs_vec_lmo}
    \begin{tabularx}{\linewidth}{|>{\raggedright\arraybackslash}X|c|c|c|}
    \hline
    Method & lmo constraint set $\mathcal D$ & lmo & Reference \\
    \hline\hline
    Normalized SGD & $l_2$-ball, $S_2$-ball & $-\eta \tfrac{g}{\norm{g}_2} = -\eta \tfrac{g}{\norm{g}_F}$ & \citep{hazan2015beyond} \\
    Momentum Normalized SGD & Ball in $l_2$, or Ball in $S_2$ & $-\eta \tfrac{g}{\norm{g}_2} = -\eta\tfrac{g}{\norm{g}_F}$ & \citep{cutkosky2020momentum}\\
    \hline
    SignSGD & Ball in Max-norm $l_\infty$ & $-\eta \sign(g)$ & \citep[Thm. 1]{bernstein2018signsgd} \\
    Signum  & Ball in Max-norm $l_\infty$ & $-\eta \sign(g)$ & \citep[Thm. 3]{bernstein2018signsgd} \\
    \hdashline
    Muon & Ball in Spectral $S_\infty$ & $-\eta UV^\top$ & \citep{jordan2024muon} \\
    \hline
    Gauss-Southwell Coordinate Descent & Ball in $l_1$ & $-\eta \sum_{i \in \argmax|g_i^t|} \sign(g_i^t)e_i$ & \citep[p.19]{shi2016primer}\\
    \hdashline 
    Neon & Ball in Nuclear $S_1$ & $-\eta u_1 v_1^\top$ & This work\\
    \hline
    \end{tabularx}
    \end{table}
\end{frame}
%------------------------------------------------------------------------------------------------------------
\begin{frame}{Understanding Dion by Thomas Pethick}
    Without momentum, Dion is simplified to
    $$
    \begin{aligned}
    \Delta &\leftarrow g + e  \\
    e &\leftarrow \Delta - \sum_{i=1}^r \sigma_i u_i v_i^\top \\
    x &\leftarrow x - \gamma \sum_{i=1}^r u_i v_i^\top
    \end{aligned}
    $$
    where $\gamma > 0$ and $\sum_{i=1}^r \sigma_i u_iv^\top_i$ is the rank-$r$ truncated SVD of $\Delta \in \Rmn$.
    \vspace{0.4em}

    \faQuestionCircle \space What will we obtain if we try $\normkfk{\mM}:=\sum_{i=1}^r{\sigma_i}$?
    \end{frame}
%-------------------------------------------------------------------------------------
\begin{frame}{Ky Fan $k$-rank $\normkfk{\mM^k}^\dagger$ and Fanions}

\begin{block}{Deriving Fanions by Recipe} $\normkfk{\mM^t}^{\dagger\dagger} = \normkfk{\mM^t}$, and $\Delta = \sum_{i=1}^k{u_i v_i^\top}$ with $\normkfk{\Delta}^\dagger = \max\{\frac{1}{k} \normn{\Delta}, \norms{\Delta}\} = \max\{\frac{1}{k} k, 1\} = 1$ delivers it:
    $$\<\mM^t, \Delta> =\<\mU \mSigma \mV^\top, \sum_{i=1}^{k} u_i v_i^\top> = \sum_{i,j=1}^{r, k}\<u_i \sigma_i v_i^\top, u_j v_j^\top> = \sum_{i=1}^{k}\sigma_i = \normkfk{\mM^t}$$
    
    Hence,
    $$\mX^{t+1} = \mX^{t} - \eta \sum_{i=1}^{k} u_i v_i^\top$$
    \end{block}
\end{frame}
%----------------------------------------
\begin{frame}{Geometric intuition: Ky Fan norms}
  \centering
  \begin{minipage}{0.46\textwidth}
      \includegraphics[height=0.75\textheight,keepaspectratio]{KyFan.pdf}
  \end{minipage}\hfill
  \begin{minipage}{0.46\textwidth}
      \includegraphics[height=0.75\textheight,keepaspectratio]{KyFanDual.pdf}
  \end{minipage}
  
  \vspace{0.3em}
  \footnotesize Ky Fan rank-2 ball and its dual in a 3D singular-value space
  \end{frame}
%-----------------------------------------------------------------------------------------------------------
\begin{frame}{Computing the $k$-rank update: Lanczos method}
    \begin{center}
    \small
    \begin{tabular}{lccc}
      \toprule
      Method & rtol & $k$ & time (s) \\
      \midrule
      Power Iterations & 0.01 & 1 & 7.7 \\
      SVDS (TRLan) & 0.01 & 1 & \textbf{0.18} \\
      PCA Low Rank (RSVD) & 0.01 & 1 & 1.15 \\
      SVDS (TRLan) & 0.01 & 10 & \textbf{0.47} \\
      PCA Low Rank (RSVD) & 0.01 & 10 & 19.4 \\
      SVDS (TRLan) & 0.01 & 100 & \textbf{1.96} \\
      PCA Low Rank (RSVD) & 0.01 & 100 & 170 \\
      \bottomrule
    \end{tabular}
    \end{center}
    \vspace{0.4em}
    \centering
    \footnotesize Comparison on a \(5000\times5000\) matrix; rtol is the relative Frobenius error of $\sum_{i=1}^{k}u_i \sigma_i v_i^\top$
  \end{frame}
%--------------------------------------------------------------------------------------------------------------
\begin{frame}{Frobeniusize the norms!}
\faExclamationCircle \space $\normf{\cdot}^\dagger = \normf{\cdot}$ is not a Ky Fan norm, so NSGD is not a Fanion.
\vspace{1em}

Let us consider a convex combination of the Ky Fan rank-$k$ norm and the Frobenius norm, which we call the F-KF-k-norm: $$\norm{\cdot}_{\mathrm{F-KF-k}} = \alpha \normkfk{\cdot} + (1-\alpha)\normf{\cdot}, \alpha \in [0,1]$$

\end{frame}
%--------------------------------------------------------------------------------------------------------------

\begin{frame}{Balls of Duals to Convex Combinations of Norms}
    \begin{lemma}\label{lemma:dual_to_conv_comb}
        Let $\norm{\cdot}_{(1)}$ and $\norm{\cdot}_{(2)}$ be norms on a finite-dimensional Euclidean space, and let $\alpha,\beta \geq 0$. Define 
        $$
        \norm{x} := \alpha \norm{x}_{(1)} + \beta \norm{x}_{(2)}.
        $$
        Then the dual unit ball of $\norm{\cdot}$ satisfies
        $$
        B_{\norm{\cdot}^\dagger} 
        = \alpha B_{\norm{\cdot}_{(1)}^\dagger} + \beta B_{\norm{\cdot}_{(2)}^\dagger},
        $$
        where $+$ denotes the Minkowski sum and $B_{\norm{\cdot}_{(i)}^\dagger}$ is the unit ball of the dual norm $\norm{\cdot}_{(i)}^\dagger$.
    \end{lemma}
    
\end{frame}
%-----------------------------------------------------------------------------------------------------------
\begin{frame}{$\normfkfk{\mM^k}^\dagger$ and F-Fanions}

    \begin{block}{Deriving F-Fanions by Recipe}
    $\normfkfk{\mM^t}^{\dagger\dagger} = \normfkfk{\mM^t}$, and $\Delta = \alpha\sum_{i=1}^k{u_i v_i^\top} + (1-\alpha)\frac{\mM^k}{\normf{\mM^k}}$ delivers it:
\begin{enumerate}
    \item $\normfkfk{\Delta}^\dagger \leq 1$ because $\sum_{i=1}^k{u_i v_i^\top}$ lies in $B_{\normkfk{\cdot}}$ and $\frac{\mM^k}{\normf{\mM^k}}$ lies in $B_{\normf{\cdot}}$, so $\Delta$ lies in the Minkowski sum
    \item $\<\mM^t, \Delta> =\alpha \<\mU \mSigma \mV^\top, \sum_{i=1}^{k} u_i v_i^\top> + (1-\alpha)\<\mM^t, \frac{\mM^t}{\normf{\mM^t}}> = \alpha \sum_{i=1}^{k}\sigma_i + (1-\alpha) \normf{\mM^t}= \normfkfk{\mM^t}$
\end{enumerate}
\vspace{1em}

        Hence,
        $$\mX^{t+1} = \mX^{t} - \eta \Big(\alpha\sum_{i=1}^{k} u_i v_i^\top + (1-\alpha)\frac{\mM^t}{\normf{\mM^t}}\Big)$$
        \end{block}
\end{frame}
%------------------------------------------------------------------------------------------------------------
% The next slides were generated by Cursor
\begin{frame}{F-Muon and F-Neon}
  Convex combinations with the Frobenius norm:
  \[
    \|\mX\|_{F*}=\alpha\|\mX\|_*+(1-\alpha)\|\mX\|_F,\qquad
    \|\mX\|_{F2}=\alpha\|\mX\|_{op}+(1-\alpha)\|\mX\|_F.
  \]
  Dual-induced updates:
  \[
    \text{F-Muon: }\mX^{k+1}=\mX^k-\eta\Big(\alpha\,\mU\mV^\top+(1-\alpha)\tfrac{\mM^k}{\|\mM^k\|_F}\Big),
  \]
  \[
    \text{F-Neon: }\mX^{k+1}=\mX^k-\eta\Big(\alpha\,u_1 v_1^\top+(1-\alpha)\tfrac{\mM^k}{\|\mM^k\|_F}\Big).
  \]
\end{frame}
%---------------------------------------
\begin{frame}{F-Muon on CIFAR-10 airbench}
    \begin{columns}[T,totalwidth=\textwidth]
      \begin{column}{0.48\textwidth}
        \includegraphics[width=\linewidth]{muon_tuned_diff_alpha.pdf}
        \centering
        \scriptsize With params tuned for Muon, as in \citep{cifar2023airbench}
      \end{column}
      \begin{column}{0.48\textwidth}
        \includegraphics[width=\linewidth]{fmuon_tuned_diff_alpha.pdf}
        \centering
        \scriptsize With params tuned for $\alpha=0.5$ F-Muon
      \end{column}
    \end{columns}
    \vspace{0.6em}
    \centering

    \footnotesize F-Muon with \(\alpha = 0.5\) matches Muon after tuning

    \faQuestionCircle \space How should we interpret cases with $\alpha > 1$?
  \end{frame}

%----------------------------------------
\begin{frame}{LMO balls on CIFAR-10}
  \begin{figure}[t]
    \includegraphics[height=0.7\textheight]{fstardual_cifar.pdf}
    \centering

    \footnotesize LMO balls for learning rates from the experiment

  \end{figure}
\end{frame}
%-----------------------------------------------------------------------------------------------
\begin{frame}{NanoGPT speedrun}
  Setup from \citet{modded_nanogpt_2024}: 1750 iterations and cross-entropy loss lower than 3.28
    \begin{itemize}
      \item Muon: {\tt lr=0.05, momentum=0.95}, final loss = 3.279
      \item F-Muon with \(\alpha=0.5\): {\tt lr=0.07, momentum=0.95}, final loss = 3.281
      \item NSGD: {\tt lr=0.07, momentum=0.96}, final loss = 3.4651
      \item F-Muon with $\alpha=0.5$: {\tt lr=0.07, momentum=0.96}, final loss = 3.2824!
    \end{itemize}
\end{frame}
%----------------------------------------
\begin{frame}{Random linear least squares: Loss and the Spectral Norm of the Gradient}
  \begin{columns}[T,totalwidth=\textwidth]
    \begin{column}{0.48\textwidth}
      \includegraphics[width=\linewidth]{simple_lls/loss_vs_iteration_50x50.pdf}
      \centering
      \scriptsize Loss vs iteration
    \end{column}
    \begin{column}{0.48\textwidth}
      \includegraphics[width=\linewidth]{simple_lls/gradient_spectral_norm_vs_iteration_50x50.pdf}
      \centering
      \scriptsize Spectral norm of gradient vs iteration
    \end{column}
  \end{columns}
  \vspace{0.3em}
  \centering
  \footnotesize Linear least squares problem for a 50 x 50 matrix. Common {\tt lr=0.01}.
\end{frame}
%----------------------------------------
\begin{frame}{Conclusion}
  \begin{itemize}
    \item Muon and NSGD are special cases of F-Fanions
    \item A combination of norm-based updates is a norm-based update
    \item No existing bound describes the superiority of the spectral norm
  \end{itemize}

  \vspace{2em}
  \textbf{For future experiments}: Exploration of the matrix and vector algorithms correspondence
\end{frame}

%----------------------------------------
\begin{frame}[allowframebreaks]{References}
  % Match article's bibliography style and entries
  \printbibliography
  %\bibliographystyle{../conf_article/icomp2024_conference}
  %\bibliography{../conf_article/icomp2024_conference}
\end{frame}

\end{document}


